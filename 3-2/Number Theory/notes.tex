\documentclass[12pt,letterpaper]{article}

%%%%%%%%%%%%
% Includes %
%%%%%%%%%%%%
\usepackage[utf8]{inputenc}
\usepackage[margin=1 in]{geometry}
\usepackage{graphicx}
\usepackage{amsmath}
\usepackage{amsthm}
\usepackage{amsfonts}
\usepackage[justification=centering]{caption}

% Formatting
\renewcommand{\baselinestretch}{1.25}

% Theorems and other necessary structures
%\theoremstyle{definition}
\newtheorem{definition}{Definition}[section] % Definition
\newtheorem{theorem}{Theorem}[section] % Big result
\newtheorem{corollary}{Corollary}[theorem] % Follows from a theorem
\newtheorem{lemma}[theorem]{Lemma} % Minor result

% New commands
\newcommand{\R}{\mathbb{R}}
\newcommand{\N}{\mathbb{N}}
\newcommand{\Z}{\mathbb{Z}}
\newcommand{\C}{\mathbb{C}}

% Title information
\title{Number Theory}
\author{2018A7PS0193P}

\begin{document}

\maketitle

\section{Fundamentals}

\subsection{Notation}

For the rest of this course, the following notation will be followed:

\begin{enumerate}
  \item $\N$ is the set of natural numbers
  \item $\Z$ is the set of integers
  \item $\mathbb{W}$ is the set of whole numbers, i.e. $\mathbb{W} = \N \cup \{0\}$
\end{enumerate}

\subsection{Induction}

Often in number theory, we use inductive proofs to prove our arguments. Induction consists of the following steps:

\begin{enumerate}
  \item Define an induction hypothesis $P(k)$
  \item Verify it works for some base case $k=b$. It is possible multiple base cases need to be verified.
  \item Assuming $P(k)$ is true, show that it implies that $P(k+1)$ is true
\end{enumerate}

Remember that $P(k)$ is a statement, not a function. You cannot multiply it by some constant or perform any operations on it.

In weak induction (like in the steps given above), we only assume that $P(k)$ is true. However in strong induction, we assume that $P(i)$ is true $\forall i \in [b,k]$, and use this to prove that $P(k+1)$ is true.

\subsection{Well Ordering Principle}

\begin{theorem}[Well Ordering Principle]
  Every non empty set of non-negative integers has a least element.
\end{theorem}

This is not true about negative integers - consider the case of infinite sets, like the set of all integers. There is no well defined least element.

This principle is equivalent to the principle of induction.

\subsubsection{Proof of Equivalence to Principle of Induction}

First, let us prove that WOP $\Rightarrow$ PMI. Let $P(n)$ be a statement that depends on $n \in \N$. Suppose that:

\begin{itemize}
  \item $P(1)$ is true
  \item $P(k)$ is true implies $P(k+1)$ is true for all $k \in N$.
\end{itemize}

We have to show that $P(n)$ is true for all $n \in N$. Let :

\[S = \{n \in \N : P(n) \text{ is true} \}\]

This means we must show that $S = \N$. Let $T := \N \symbol{92} S$, i.e. $T$ is the complement. Let as assume that $S \neq \N$.

By WOP, $T$ has a least element, say $m$. Note that $m \geq 2$ since $1 \in S$. Then, $m-1 \notin T$ and $m-1 \in S$. As such, $P(m-1)$ must be true! However, by our initial assumptions, that would mean $P(m)$ is true as well, so $m \in S$. This creates a contradiction, since $m \in T$. Hence, $S = \N$.


\subsection{Binomial Theorem}

\begin{theorem}[Binomial Theorem]
  Let $x,y \in \C$ and let $n \in \N$, then 
  \[(x+y)^n = \sum_{k=0}^{n} {n \choose k} x^ky^{n-k}\]
\end{theorem}


\begin{corollary}
  \[\sum_{k=0}^n {n \choose k} = 2^n\]
\end{corollary}

\begin{lemma}[Pascal's Identity]
  \[{n \choose k} = {n-1 \choose k-1} + {n-1 \choose k}\]
\end{lemma}

\begin{lemma}
  \[\sum_{k=0}^{\left \lfloor n/2 \rfloor \right} {n-k \choose k} = F_n\]
\end{lemma}

\subsection{Pigeonhole Principle}

\begin{theorem}
  If $n$ items are put into $m$ containers, with $n > m$, then at least one container must contain more than one item.
\end{theorem}

\section{The Division Algorithm}

\begin{theorem}
  Let $a,b \in \Z$ with $b > 0$. Then, there exist unique integers $q$ and $r$ such that $a = bq + r$, $r \in [0,b)$.
\end{theorem}

\noindent \textbf{Proof:}

Let $S = \{a - bn : n \in Z, a-bn \geq 0\}$. This set is always non-empty:
\begin{itemize}
  \item If $a \geq 0$, then $a \in S$
  \item If $a<0$, then if $n = a$, we have $a-ab \in S$ since $b \geq 1$.
\end{itemize}

By WOP, $S$ has a least element, say $r$. So, there exists $q \in Z$ such that $r = a-bq$. Since $r \in S$, we have $r \geq 0$.

Suppose $r \geq b$. Then:

\[a-b(q+1) = a-bq-b = r-b \geq 0\]
\[ \Rightarrow a-b(q+1) \in S\]
\[\Rightarrow r-b \in S\]

However, $r-b < r$, and $r$ is the least element! This gives us a contradiction. So, $r < b$.

As such, we have proved the existence of this solution. Now we must prove it's uniqueness.

Suppose there exists $p,r,q',r'$, such that:
\[a = bq+r, 0 \leq r < b\]
\[a = bq'+r', 0 \leq r' < b\]

Assume WLOG $q \geq q'$. Now,
\[r'-r = b(q-q')\]
If $q > q'$, then $r'-r \geq b$. However, $r'-r < b$. So, this is a contradiction, and $q' = q$. The solution must be unique.

\begin{definition}
  If $a,b \in \Z$, we say that $a$ divides $b$ if $b=ak$ for some $k \in \Z$. This is denoted by $a|b$
\end{definition}

Some properties of division are:

\begin{itemize}
  \item If $a|b$, then $\pm a | \pm b$
  \item If $a|b$ and $b|c$ then $a|c$ (Transitivity)
  \item If $a|b$ and $a|c$ then $a|bx+cy$ (Linear Combination)
  \item If $a|b$ and $b \neq 0$, then $|a| \leq |b|$ (Bounds by divisibility)
  \item $a|b$ and $b|a$, then $b = \pm a$.
\end{itemize}

\end{document}
