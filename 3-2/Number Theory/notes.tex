\documentclass[12pt,letterpaper]{amsbook}

% Formatting packages
\usepackage[utf8]{inputenc}
\usepackage[margin=1 in]{geometry}
\usepackage{parskip}
\usepackage[hidelinks]{hyperref}

% Picture packages
\usepackage{graphicx}
\usepackage[justification=centering]{caption}

%AMS packages
\usepackage{amsmath,amsthm,amsfonts,amssymb,mathtools}

% Formatting
\renewcommand{\baselinestretch}{1.25}

% Theorems and other necessary structures
\usepackage{mdframed}
\mdfsetup{skipabove=1em,skipbelow=0em}
\theoremstyle{definition}
\newmdtheoremenv[nobreak=true]{theorem}{Theorem}[chapter] % Big result
\newmdtheoremenv[nobreak=true]{corollary}{Corollary}[theorem] % Follows from a theorem
\newmdtheoremenv[nobreak=true]{lemma}[theorem]{Lemma} % Minor result
\newtheorem{definition}{Definition} % Definition
\newtheorem*{remark}{Remark}
\newtheorem*{exercise}{Exercise}

\newenvironment{solution}
  {\renewcommand\qedsymbol{$\blacksquare$}\begin{proof}[Solution]}
  {\end{proof}}

% New commands
\newcommand{\R}{\mathbb{R}}
\newcommand{\N}{\mathbb{N}}
\newcommand{\Z}{\mathbb{Z}}
\newcommand{\C}{\mathbb{C}}

% Title information
\title{Number Theory}
\author{2018A7PS0193P}

\begin{document}

\maketitle

\tableofcontents

\chapter{Fundamentals}

\section{Notation}

For the rest of this course, the following notation will be followed:

\begin{enumerate}
  \item $\N$ is the set of natural numbers
  \item $\Z$ is the set of integers
  \item $\mathbb{W}$ is the set of whole numbers, i.e. $\mathbb{W} = \N \cup \{0\}$
\end{enumerate}

\section{Induction}

Often in number theory, we use inductive proofs to prove our arguments. Induction consists of the following steps:

\begin{enumerate}
  \item Define an induction hypothesis $P(k)$
  \item Verify it works for some base case $k=b$. It is possible multiple base cases need to be verified.
  \item Assuming $P(k)$ is true, show that it implies that $P(k+1)$ is true
\end{enumerate}

Remember that $P(k)$ is a statement, not a function. You cannot multiply it by some constant or perform any operations on it.

In weak induction (like in the steps given above), we only assume that $P(k)$ is true. However in strong induction, we assume that $P(i)$ is true $\forall i \in [b,k]$, and use this to prove that $P(k+1)$ is true.

\begin{exercise}
  Prove that the principle of strong induction is true given that the principle of weak induction is true.
\end{exercise}
\begin{solution}
  Let us assume that $P(1),...,P(b)$ is true. If $P(1),...,P(k)$ are true for some $k \geq b$, then $P(k+1)$ is true. Then, we must show that $P(n)$ is true for all $n \geq 1$. 

  Let $Q(n)$ be the statement that $P(1),...P(n)$ are true. Of course, in the base case, $Q(1)$ is true. Let $Q(k)$ be true, where $K \geq 1$. This means that $P(1),...P(k)$ is true, so $P(k+1)$ must be true. Hence, $Q(k+1)$ is true.

  So, by Weak induction, $Q(n)$ is true $\forall n \geq 1$, which implies that $P(n)$ is true $\forall n \geq 1$.
\end{solution}

\section{Well Ordering Principle}

\phantom{}

\begin{theorem}[Well Ordering Principle]
  Every non empty set of non-negative integers has a least element.
\end{theorem}

This is not true about negative integers - consider the case of infinite sets, like the set of all integers. There is no well defined least element.

\begin{lemma}
The well ordering principle is equivalent to the principle of mathematical induction.
\end{lemma}

\begin{proof}
First, let us prove that WOP $\Rightarrow$ PMI. Let $P(n)$ be a statement that depends on $n \in \N$. Suppose that:

\begin{itemize}
  \item $P(1)$ is true
  \item $P(k)$ is true implies $P(k+1)$ is true for all $k \in N$.
\end{itemize}

We have to show that $P(n)$ is true for all $n \in N$. Let :

\[S = \{n \in \N : P(n) \text{ is true} \}\]

This means we must show that $S = \N$. Let $T := \N \symbol{92} S$, i.e. $T$ is the complement. Let as assume that $S \neq \N$.

By WOP, $T$ has a least element, say $m$. Note that $m \geq 2$ since $1 \in S$. Then, $m-1 \notin T$ and $m-1 \in S$. As such, $P(m-1)$ must be true! However, by our initial assumptions, that would mean $P(m)$ is true as well, so $m \in S$. This creates a contradiction, since $m \in T$. Hence, $S = \N$.

Now, let us prove that PMI $\Rightarrow$ WOP.

Consider the statement $P(n)$ that every non empty set of non-negative integers of size $n$ has a least element. It is clear that the base case $P(1)$ is true. Now, let us assume that $P(k)$ is true - what can we say about $P(k+1)$. When we insert an element, we have two cases:

\begin{enumerate}
  \item The inserted element is less than the least element. In this case, there is a new least element, and $P(k+1)$ is true.
  \item The inserted element is not less than the least element. In this case, the least element is the same, and $P(k+1)$ is true.
\end{enumerate}

Hence, by PMI, we can say that $P(n)$ is true $\forall n \in N$, i.e., WOP is true.

Since PMI $\Rightarrow$ WOP and WOP $\Rightarrow$ PMI, PMI $\Leftrightarrow$ WOP.
\end{proof}

\section{Binomial Theorem}

\phantom{}

\begin{theorem}[Binomial Theorem]
  Let $x,y \in \C$ and let $n \in \N$, then 
  \[(x+y)^n = \sum_{k=0}^{n} {n \choose k} x^ky^{n-k}\]
\end{theorem}


\begin{corollary}
  \[\sum_{k=0}^n {n \choose k} = 2^n\]
\end{corollary}

\begin{lemma}[Pascal's Identity]
  \[{n \choose k} = {n-1 \choose k-1} + {n-1 \choose k}\]
\end{lemma}

\begin{lemma}
  \[\sum_{k=0}^{\left \lfloor n/2 \rfloor \right} {n-k \choose k} = F_n\]
\end{lemma}

\phantom{}

\section{Pigeonhole Principle}

\phantom{}

\begin{theorem}
  If $n$ items are put into $m$ containers, with $n > m$, then at least one container must contain more than one item.
\end{theorem}

\chapter{Division}

\section{Division Algorithm}

\phantom{}

\begin{theorem}
  Let $a,b \in \Z$ with $b > 0$. Then, there exist unique integers $q$ and $r$ such that $a = bq + r$, $r \in [0,b)$.
\end{theorem}

\begin{proof}

Let $S = \{a - bn : n \in Z, a-bn \geq 0\}$. This set is always non-empty:
\begin{itemize}
  \item If $a \geq 0$, then $a \in S$
  \item If $a<0$, then if $n = a$, we have $a-ab \in S$ since $b \geq 1$.
\end{itemize}

By WOP, $S$ has a least element, say $r$. So, there exists $q \in Z$ such that $r = a-bq$. Since $r \in S$, we have $r \geq 0$.

Suppose $r \geq b$. Then:

\[a-b(q+1) = a-bq-b = r-b \geq 0\]
\[ \Rightarrow a-b(q+1) \in S\]
\[\Rightarrow r-b \in S\]

However, $r-b < r$, and $r$ is the least element! This gives us a contradiction. So, $r < b$.

As such, we have proved the existence of this solution. Now we must prove it's uniqueness.

Suppose there exists $p,r,q',r'$, such that:
\[a = bq+r, 0 \leq r < b\]
\[a = bq'+r', 0 \leq r' < b\]

Assume WLOG $q \geq q'$. Now,
\[r'-r = b(q-q')\]
If $q > q'$, then $r'-r \geq b$. However, $r'-r < b$. So, this is a contradiction, and $q' = q$. The solution must be unique.

\end{proof}

\begin{definition}
  If $a,b \in \Z$, we say that $a$ divides $b$ if $b=ak$ for some $k \in \Z$. This is denoted by $a|b$
\end{definition}

Some properties of division are:

\begin{itemize}
  \item If $a|b$, then $\pm a | \pm b$
  \item If $a|b$ and $b|c$ then $a|c$ (Transitivity)
  \item If $a|b$ and $a|c$ then $a|bx+cy$ (Linear Combination)
  \item If $a|b$ and $b \neq 0$, then $|a| \leq |b|$ (Bounds by divisibility)
  \item $a|b$ and $b|a$, then $b = \pm a$.
\end{itemize}

\begin{exercise}
  Prove that $x^a-1|x^b-1 \Leftrightarrow a|b$.
\end{exercise}
\begin{solution}
  First, let us prove that if $a|b$, then $x^a-1|x^b-1$. Let $b=qa$. Then,
  \[x^b-1 = (x^a)^q-1^q = (x^a-1)((x^a)^{q-1} + \cdots + x^a + 1)\]
  So, $x^b-1 | x^a-1$.
  Now to prove the converse. Let $b=aq+r$. Assume $a \nmid b$, then $0 < r < a$. Then,
  \[x^b-1 = x^b - x^r + x^r - 1 = x^r(x^{aq} - 1) + x^r-1\]
  $x^a-1|x^{qa}-1$, so $x^r-1$ is the remainder. Since $r < a$, $x^a-1 \nmid x^r-1$. This would mean that $x^a-1 \nmid x^b-1$, which is a contradiction. So, $r = 0$. Hence proved.
\end{solution}

\section{Base b representations}

\phantom{}

\begin{theorem}
  Let $b \in \N$   with $b \geq 2$. Then every positive integer can be expressed uniquely as 
  \[N = a_kb^k + ...+a_1b+a_0\]
  where $k \geq 0, a_k \neq 0$ and $0 \leq a_i < b$ for $i = 0,...k$. This is denoted by $N = (a_k,...a_1a_0)_b$
\end{theorem}

\begin{proof}
  By the division algorithm, there exist unique integers $q_0$ and $a_0$ such that:
  \[N = q_0b + a_0, a_0 \in [0,b)\]
  Note that $q_0 < N$. If $q_0 \neq 0$ we apply the division algorithm again to find unique integers $q_1$ and $a_q$ such that:
  \[q_0 = q_1b + a_1, a_1 \in [0,b)\]
  Then,
  \[N = (q_1b+a_1)b+a_0 = q_1b^2 + a_1b + a_0\]
  We continue till we get a quotient $q_k = 0$. This will terminate since $q_k < ... < q_2 < q_1 < q_0 < N$, forming a decreasing sequence of non-negative integers and eventually reaching zero. From this, we get:
  \[N = a_kb^k + ... + a_1b + a_0\]
  Hence, the solution always exists.

  Suppose $N$ has two distinct expansions. We can write it as:
  \begin{align*}
    N &= a_kb^k + ... + a_1b + a_0 \\
      &= c_kb^k + ... + c_1b + c_0
  \end{align*}
  where $0 \leq a_i,c_j < b$ for all $i,j$. Let $d_i = a_i-c_i$. Then, $\sum_{i=0}^k d_ib^i = 0$. The $d_i$ cannot all be zero as the two expansions are assumed distinct. Let $j$ be the least integer, $0 \leq j \leq k$, such that $d_j \neq 0$. Then, $\sum_{i=j}^k d_ib^i = 0$. Dividing by $b^j$, we find that $\sum_{i=j}^k d_ib^{i-j} = 0$. Thus,
  \[d_j + b \left( \sum_{i = j+1}^{k} d_i b^{i-j-1} \right) = 0\]
  This implies that the $b | d_j$ and since $d_j \neq 0$, we get that $b = |b| \leq |d_j|$. However, $|d_j| < b$. Hence, we have a contradiction, and the two expansions cannot be distinct.
  Hence, the solution is also always unique.
\end{proof}

\begin{lemma}
  If $N = (a_k...a_1a_0)_b$, then:
  \[bN = (a_k...a_1a_00)_b\]
  \[\Bigl \lfloor \frac{N}{b} \Bigr \rfloor = (a_k...a_1)_b\]
\end{lemma}

This is a trivial result, which can be thought of as a left or right bitwise shift.

\begin{lemma}[Particular case of Legendre's formula]
  Let $n \in \N$ and let $e$ denote the highest power of 2 dividing $n!$. Then
  \[e = \sum_{k=1}^{\infty} \Bigl \lfloor \frac{n}{2^k} \Bigr \rfloor\]
  This is always a finite sum. This can alternatively expressed as, if $n = (a_k...a_1a_0)_2$, then:
  \[e = n - (a_k + ... + a_1 + a_0)\]
\end{lemma}
\begin{proof}
  It is clear that $e$ is the sum of the no. of positive multiples of $2^i$ which are $\leq n$, for all $i$. So, this can be calculated by:
  \begin{align*}
    e = \sum_{k=1}^{\infty} \Bigl \lfloor \frac{n}{2^k} \Bigr \rfloor    
  \end{align*}
\end{proof}

Thus, if $r$ denotes the number of ones in the binary expansion of $n$, then $2^{n-r}$ is the highest power of 2 dividing $n!$. Further,
\begin{itemize}
  \item $2^n \nmid n!$ for $n \in \N$
  \item $2^{n-1} | n!$ if and only if $n$ is a power of 2.
\end{itemize}

\chapter{Properties of Numbers}

\section{Prime and Composite Numbers}

\begin{definition}
  A positive integer $p > 1$ is called prime if its only positive divisors are 1 and $p$. A positive integer which is not prime is called composite.  
\end{definition}

The number 1 is neither prime nor composite.

\begin{lemma}
  Every integer $n \geq 2$ has a prime factor.
\end{lemma}

\begin{proof}
  Let $P(n)$ be the statement that $n$ has a prime factor. Then $P(2)$ is true, since 2 is a prime factor of 2. Let $k \geq 2$. Assume $P(2)...P(k)$ are true.

  If $k+1$ is prime, then $k+1$ is a prime factor of itself. So $P(k+1)$ is true.

  If $k+1$ is composite, then there exists $d \in [2,k]$ such that $d | k+1$. By the induction hypothesis, $d$ has a prime factor $p$. Since $p|d$ and $d|k+1$, $p|k+1$. So $p$ is a prime factor of $k+1$, and $P(k+1)$ is true. By PSI, $P(n)$ is true for all $n \geq 2$.
\end{proof}

\begin{theorem}[Euclid]
  There are infinitely many primes.  
\end{theorem}

\begin{proof}
  Suppose there are finitely many primes $p_1,...,p_k$. Let
  \[N = p_1...p_k + 1\]
  Since $N \geq 2$, it must have a prime factor. Hence, there exists $i \in [1,k]$ such that $p_i | N$. Since $p_i|N$ and $p_i|p_1p_2...p_k$, we get that $p_i|N-p_1p_2...p_k$, i.e., $p_i|1$. However, $p_i \geq 2$, which gives us a contradiction. So, there must be infinitely many primes.
\end{proof}

\begin{exercise}
  For $n \geq 1$, let $p_n$ be the $n$th prime. Prove that
  \[p_n \leq 2^{2^{n-1}}\]
\end{exercise}

\begin{solution}
  Let $P(n)$ be the statement that $p_n \leq 2^{2^{n-1}}$. It is clear that this is true for the base case $P(1)$. Let us assume that $P(1),...,P(k)$ is true for $k \geq 1$.
  We observed in Euclid's proof that $p_1...p_k+1$ is not divisible by any of $p_1...p_k$. Hence if $p_i$ denotes a prime factor of $p_1...p_k+1$, then $i \geq k+1$.
  \[p_{k+1} \leq p_i \leq p_1...p_k+1\]
  Using the inductive hypothesis, we find that
  \begin{align*}
    p_{k+1} &\leq p_1...p_k+1 \leq 2.2^2.2^{2^2}...2^{2^{k-1}}+1 \\
            &= 2^{\sum_{j=0}^{k-1} 2^j}+1 = 2^{2^k-1}+1 \leq 2^{2^k}
  \end{align*}
  So, $P(k+1)$ is true. So, by PSI, the result has been proven.
\end{solution}

\begin{definition}
  The product of the first $n$ prime numbers is called the $n^{th}$ primorial and is denoted by $p_n\#$.
\end{definition}

\begin{definition}
  Euclid numbers are integers of the form $E_n = p_n\# +1$. 
\end{definition}

All Euclid numbers are not primes - $E_6$ is not a prime!

\begin{theorem}
  Every composite number $n$ has a prime factor $\leq \lfloor \sqrt{n} \rfloor$
\end{theorem}

\begin{proof}
  Since $n$ is composite, there exists integers $k,l \in (1,n)$   such that 
  \[n = kl\]
  If $k > \sqrt{n}$ and $l > \sqrt{n}$ then $kl > n$, which is false. So, one of them must be less than or equal to $\sqrt{n}$.
\end{proof}

So, if $n > 1$ has no prime factors $\leq \lfloor \sqrt{n} \rfloor$, then $n$ is prime. We can use this as a test of primality.

It is faster to do this using the Sieve of Eratosthenes. Using this, we can test primality of the first $n$ integers in $O(n \log{\log{n}})$ instead of $O(n\sqrt{n})$. This is a pretty well known algorithm so it's left to the reader to see it on cp-algorithms.

\begin{theorem}
  There is no non-constant polynomial $f(x)$ with integer coefficients such that $f(n)$ is prime for all integer $n$.  
\end{theorem}

\begin{proof}
  Suppose such a polynomial $f(x)$ exists:
  \[f(x) = a_kx^k + ... + a_1x + a_0, k \geq 1, a_k \neq 0\]
  Let $b \in \Z$. Then $f(b)$ is a prime number, say $p$. Let $t \in \Z$. We have:
  \begin{align*}
    f(b+tp) &= a_k(b+tp)^k + ... + a_1(b+tp) + a_0 \\
            &= (a_kb^k + ... a_1b + a_0) + p \cdot g(t) \\
            &= f(b) + p \cdot g(t) = p(1+g(t))
  \end{align*}
  where $g(t)$ is a polynomial in $t$. Since $p | f(b+tp)$ and it must be prime, so $f(b+tp) = p$. This implies that $f$ assumes the value $p$ infinitely many times. This is a contradiction, since a polynomial of degree $k$ cannot assume the same value > $k$ times.
\end{proof}

\section{Prime Counting function}
Let $x$ be a positive real number. We define :
\[\pi (x) = \sum_{p \leq x} 1\]
where $p$ denotes a prime. So $\pi (x)$ counts the number of primes $\leq x$. This is called the prime counting function.

\begin{theorem}[Prime Number Theorem]
  \[\lim_{x \rightarrow \infty} \frac{\pi (x)}{x/\log{x}} = 1\]
\end{theorem}

This essentially states that $\pi (x) \sim \frac{x}{\log{x}}$. The proof is too complicated to be covered in this course.

\section{Gaps between Primes}

The following lemma states that we can find a gap between primes of any arbitrary length.

\begin{lemma}
  For every $n \in \N$, there are $n$ consecutive integers that are all composite.  
\end{lemma}

\begin{proof}
  Consider the numbers:
  \[(n+1)! + 2, (n+1)! + 3, ... ,(n+1)!+(n+1)\]
  It is clear that for $n \geq 1$, $2|(n+1)!+2$. However, $(n+1)! + 2 \neq 2$. So, $(n+1)!+2$ cannot be prime, and must be composite. We can extend this to each of the given numbers, and prove that they are all composite.
\end{proof}

\begin{definition}
  A pair $(p,q)$ of primes with $p < q$ is called a twin prime pair if $q-p = 2$.  
\end{definition}

It is unknown how many twin primes exist. It is conjectured that there are infinitely many twin primes, but this has not yet been proved.

\begin{theorem}[Bertrand's Postulate]
  For every integer $n \geq 2$, there is always at least one prime between $n$ and $2n$.  
\end{theorem}

This was verified by Bertrand but proved by Chebyshev. It is sometimes called Chebyshev's theorem. The proof of this result goes beyond the scope of this course.

\begin{remark}
  Do not use this result unless mentioned that we can, in the exam .
\end{remark}

\begin{exercise}
  Using Bertrand's postulate, prove that for $n \geq 2$:
  \[p_n < 2^n\]
\end{exercise}

\begin{solution}
  Consider the statement $P(n)$ that $p_n < 2^n$. This is true for the base case that $P(2)$. Now assume that $P(k)$ is true. This means:
  \[p_k < 2^k\]
  From Bertrand's postulate,
  \[k < p_k < 2k\]
  \[k+1 \leq p_k \]
  We also know from Bertrand's postulate that:
  \[k+1 < p_{k+1} < 2(k+1)\]
  \[p_{k+1} < 2p_k\]
  So, from the induction hypothesis,
  \[p_{k+1} < 2.2^k\]
  \[p_{k+1} < 2^{k+1}\]
  Hence, $P(k) \Rightarrow P(k+1)$. From PMI, $P(n)$ is true $\forall n \geq 2$.
\end{solution}
\begin{exercise}
  Prove that if $2^{m}+1$ is prime, then $m = 2^n$ for some $n$.
\end{exercise}
\begin{solution}
  Here, we use the following lemma - if $k$ is odd, then $x^k+1$ is divisible by $x+1$.
  Suppose that $m$ has an odd factor $k$. Then, we can express $m$ as $kp$. So,
  \[2^{kp}+1 = (2^{p})^k + 1\]
  From our lemma, this is divisible by $2^p+1$ which is a number other than 1 and itself. This means $2^m+1$ cannot be prime if it has an odd factor, and hence $m$ must be a power of 2.
\end{solution}

\section{Fermat Numbers}

\begin{definition}
  Fermat numbers are $f_n$ such that:
  \[f_n = 2^{2^n}+1\]
\end{definition}

\begin{lemma}[Recursive definition]
  \[f_n = f_{n-1}^2 - 2f_{n-1} + 2\]  
\end{lemma}

This result is obvious from expanding the RHS, so the proof is not given here.

\begin{exercise}
  Prove that $f_n$, $n \geq 2$, all end in 7.
\end{exercise}
\begin{solution}
  Let $P(n)$ be the statement that $f_n$ ends in 7. This is true for our base case $P(2)$. Let us assume that $P(k-1)$ is true, i.e. $f_k \mod 10 = 7$. So, by the recursive definition:
  \begin{align*}
    f_k &= f_{k-1}^2 - 2f_{k-1} + 2 \mod 10 \\
        &= 7^2 - 2* 7 + 2 \mod 10 \\ 
        &= 7 \mod 10
  \end{align*}
  So, by PMI, $P(n)$ is true for all $n \geq 2$.
\end{solution}

\begin{lemma}[Duncan's Identity]
  \[f_0f_1...f_{n-1} = f_n - 2\]  
\end{lemma}
\begin{proof}
  Let $P(n)$ be the statement that this is true for $f_n$. This is clearly true for the base case $P(1)$. Let us assume $P(k)$ is true, i.e.
  \[f_0f_1...f_{k-1} = f_k - 2\]
  \[f_0f_1...f_k = f_k(f_k-2) = f_k^2-2f_k = f_{k+1}-2\]
  The above result comes from the recursive definition. Since $P(k+1)$ follows from $P(k)$, by PMI, Duncan's identity is true.
\end{proof}

\begin{theorem}
  Every prime factor of $f_n$, $n \geq 2$, is of the form $k \cdot 2^{n+2} + 1$.
\end{theorem}
\begin{proof}
  To be discussed later in the course.  
\end{proof}

This theorem can be helpful to quickly find the primality of $f_n$. For instance, $f_4$ is prime - we can see this by checking all the numbers of the form $2^6 k+1$ which are less than $\sqrt{f_4}$. This cuts down the search space and makes primality checking faster.

\section{Fibonacci Numbers}

\begin{definition}
  Fibonacci numbers are numbers of the form:
  \[F_n = F_{n-1} + F_{n-2}\]
  where $F_1 = 1, F_2 = 1$
\end{definition}

\begin{lemma}
  \[\sum_{i=1}^{k} F_i = F_{k+2}-1\]
\end{lemma}

\begin{lemma}[Cassini's Formula]
  \[F_{n-1}F_{n+1} - F_n^2 = (-1)^n, n \geq 2\]
\end{lemma}

The proofs of lemmas 3.11 and 3.12 come directly from induction, so I am not discussing the proof here.

\begin{lemma}
  \[F_{n+m} = F_m F_{n+1} + F_{m-1} F_n, m \geq 2, n \geq 1\]
\end{lemma}
\begin{proof}
  This is a non-trivial case of induction. Let us fix $m \in \N$, and do induction on $n$. For $n=1$, the RHS is:
  \[F_{m-1}F_1 + F_m F_2 = F_{m-1} + F_m = F_{m+1}\]
  This is also true for $n=2$.
  Assume that the result is true for $k=3,4,...,n$. We want to show that the result is true for $k=n+1$. For $k=n-1$, 
  \[F_{m+n-1} = F_{m-1}F_{n-1} + F_{m}F_n\]
  For $k=n$,
  \[F_{m+n} = F_{m-1}F_n + F_{m}F_{n+1}\]
  Add both sides, we get:
  \[F_{m+n-1} + F_{m+n} = F_{m+n+1} = F_{m-1}F_{n+1} + F_{m}F_{n+2}\]
  Hence Proved.
\end{proof} 
\section{Lucas Numbers}

\begin{definition}
  Lucas numbers are numbers $L_n$ such that:
  \[L_n =L_{n-1} + L_{n-2}\]
  where L_1 = 1, L_2 = 3.
\end{definition}

\begin{theorem}[Binet's formulas]
  Let $\alpha = \frac{1+\sqrt 5}{2}$ and $\beta = \frac{1-\sqrt 5}{2}$.
  \[F_n = \frac{\alpha^n-\beta^n}{\alpha-\beta}\]
  \[L_n = \alpha^n + \beta^n\]
\end{theorem}
\begin{proof}
 TODO  
\end{proof}

\chapter{Greatest Common Divisor and Least Common Multiple}

\section{Greatest Common Divisor}

\begin{definition}
  Let $a,b \in \Z$, not both zero. The greatest common divisor of $a$ and $b$ is the positive integer $d$ such that:
  \begin{itemize}
    \item $d|a$ and $d|b$
    \item If $c$ is a positive integer such that $c|a$ and $c|b$, then $c \leq d$.
  \end{itemize}
  This is generally denoted by $(a,b)$.
\end{definition}

The GCD of two non-zero numbers always exists and is unique. Observe that $(a,b) = (a,-b) = (-a,b) = (-a,-b)$. If $a \neq 0$, then $(a,0) = |a|$.

\begin{definition}
  $a,b \in \Z$ are relatively prime (coprime) if $(a,b) = 1$.
\end{definition}

\begin{exercise}
  Prove that $(F_n,F_{n+1}) = 1$ for $n \geq 1$.  
\end{exercise}
\begin{solution}
  From Cassini's formula, we know that:
  \[F_{n-1}F_{n+1} - F_n^2 = (-1)^n\]
  Let $d = (F_n,F_{n+1})$. Since $d | F_n$ and $d|F_{n+1}$, we have $d|F_{n-1}F_{n+1}-F_n^2$. So $d | (-1)^n$. This means that $|d| \leq |(-1)^n|$. Since $d$ is a positive number, $d = 1$.
\end{solution}

\begin{exercise}
  Prove that $(f_m,f_n) = 1$ for distinct non-negative $m,n$.
\end{exercise}
\begin{solution}
  Suppose $m < n$. Let $d = (f_m,f_n)$. Since $d|f_m$ and $d|f_n$, $d|f_n-(f_0...f_m..f_{n-1})$. From Duncan's identity, this implies that $d|2$. Since $d> 0, $, $d = 1$ or $d=2$. However, Fermat numbers are always odd - so $d \neq 2$. This implies that $d = 1$.
\end{solution}

\begin{theorem}
  Let $a,b \in \Z$, not both zero. Then there exist integers $x_0, y_0$ such that:
  \[(a,b) = ax_0 + by_0\]
\end{theorem}
\begin{proof}
  Consider the set $S = \{ax+by>0 : a,b \in \Z \}$. Let $d = \min S$.
  Suppose that $d$ does not divide $a$. Then by division algorithm:
  \[a=qd+r\]
  \[qd = a-r\]
  \[q(ax+by) = a-r\]
  \[r = a(1-qx)-bqy\]
  So, $r$ is a linear combination of $a$ and $b$, and since $r > 0$, $r \in S$. From division algorithm, $r < d$, which contradicts the fact that $d  = \min S$. So, by contradiction, $d$ divides $a$ (and by similiar argument, $d$ divides $b$).
  We also know that any common divisor of $a$ and $b$ must divide $d$. This is obvious since if $a = uc$ and $b = vc$, then $d = ax+by = c(ux+vy)$, so $c|d$.
  From these two facts, it is clear that $d$ is the GCD, and is of the form $ax+by$.
\end{proof}

\begin{exercise}
  Let $a,b \in \N$. If $b = aq+r$, then $(a,b) = (a,r)$.
\end{exercise}
\begin{solution}
  Let $d = (a,b)$ and $e = (b,r)$. We need to show that $d = e$. Since $d|a$ and $d|b$, $d|a-bq=r$. So $d$ is a common divisor of $b$ and $r$. Hence $d \leq e$. Similarly, as $e|b$ and $e|r$, $e$ divides $bq+r = q$. Thus $e$  is a common divisor of $a$ and $b$, so $e \leq d$. Thus, $e = d$.
\end{solution}

\begin{exercise}
  Let $a,b,c \in \N$. Prove that $(ac,bc) = c(a,b)$.  
\end{exercise}
\begin{solution}
  Let $d = (a,b)$. Then $d|a$ and $d|b$, so $d|ca$ and $d|cb$.  There exist integers $x,y$ such that:
  \[d = ax+by\]
  \[cd = (ac)x + (bc)y\]

  If $e$ is a positive integer such that $e | ac$ and $e|bc$, then $e|(ac)x+(bc)y$, i.e. $e|cd$. So, $cd$ is the GCD of $ac$ and $bc$.
\end{solution}

\begin{theorem}
  Let $a,v \in \Z$, not both zero. Then $(a,b) = 1$ if and only if there exist integers $x_0,y_0$ such that $ax_0 + by_0 = 1$.  
\end{theorem}
\begin{proof}
  If $(a,b) = 1$, there must exist $x_0,y_0$ such that $ax_0+by_0$ = 1, from Theorem 4.1.
  Conversely, suppose there exists $x_0,y_0 \in \Z$, such that 
  \[ax_0 + by_0 = 1\]
  Let $d = (a,b)$. Then $d|ax_0+by_0$, which means that $d|1|$. Since $d \in \N$, $d=1$.
\end{proof}

\begin{corollary}
  Let $d = (a,b)$. Then, $(\frac{a}{d}, \frac{b}{d}) = 1$
\end{corollary}

\begin{corollary}
  If $(a,b) = 1$, and $a$ and $b$ both divide $c$, then $ab|c$.  
\end{corollary}

% I know, it's called a lemma and now we are calling it a theorem!
\begin{theorem}[Euclid's Lemma]
  If $a|bc$ and $(a,b) = 1$, then $a|c$.  
\end{theorem}
\begin{proof}
  \[ax+by = 1\]
  \[acx+bcy=c\]
  Since $a|acx$ and $a|bcy$, so $a|c$.
\end{proof}

\begin{exercise}
  Let $m, n \in \N, m > 2$. If $F_m|F_n$, prove that $m|n$.
\end{exercise}
\begin{proof}
  We know that:
  \[F_n = F_{n-m}F_{m-1} + F_{n-m+1}F_m\]
  Since $F_m|F_n$ and $F_m|F_{n-m+1}F_m$, we get $F_m|F_n-F_{n-m+1}F_m$ and hence $F_m|F_{n-m}F_{m-1}$. But, we also know that $(F_m,F_{m-1}) = 1$. By Euclid's Lemma, $F_m|F_{n-m}$.
  
  From the division algorithm, let $n = mq+r$. Suppose $r > 0$. From our previous result, $F_m|F_{n-m}$, so $F_m|F_{n-2m}...F_m|F_r$. This means that $F_m \leq F_r$. But $r < m$, so $F_r < F_m$. This is a contradiction. So $r = 0$. Hence proved.
\end{proof}

\begin{definition}
  Let $n \geq 2$ and let $a_1,...a_n \in \Z$, not all zero. The GCD of $a_1,...,a_n$ is the largest positive integer that divides each $a_i$. This is denoted by $a_1,...,a_n$.  
\end{definition}

This has the following properties:

\begin{itemize}
  \item $(a_1,a_2...a_n)$ is the least positive integer that is a linear combination of $a_1...a_n$.
  \item $(a_1,...,a_n) = ((a_1,...,a_{n-1}),a_n)$
  \item If $d | a_1...a_n$ and $(d,a_i) = 1$ for all $i \in [1,n-1]$, then $d|a_n$.
  \item If $a_1,...,a_n$ are pairwise relatively prime, then $(a_1,...,a_n) = 1$.
\end{itemize}

\section{Euclidean Algorithm}

We are given $a,b \in \Z$, not both zero, and want to compute $(a,b)$. The algorithm to do this works as follows:

\begin{enumerate}
  \item If $a$ or $b$ are negative, replace with their absolute value.
    \item If $a > b$, then swap $a$ and $b$.
    \item If $a = 0$, then $(a,b) = b$.
    \item If $a >0$, write $b = aq+r$. Then, 
      \[(a,b) = (r,a)\]
      Go to step 3 with $a = r$ and $b=a$ respectively.
\end{enumerate}

This is called the Euclidean Algorithm. 

To express $(a,b)$ as a linear combination of $a$ and $b$ where $0 \leq a \leq b$, we create a table with four columns with headings $x,y,r,q$. We denote the rows as $R_{-1}, R_0, R_1, ... R_{i-1}$ and the entries in $R_i$ as $x_i,y_i,r_i,q_i$. Suppose we have filled $R_{i-1}$ for some $i \geq 1$. To fill $R_i$, we first compute $q_i$, which is the quotient obtained on dividing $r_{i-2}$ by $r_{i-1}$. Next, $R_i = R_{i-2} - q_iR_{i-1}$. This is known as the Extended Euclidean Algorithm.

\begin{theorem}[Lame's theorem]
  Let $b \geq a \geq 2$. The number of divisions required to compute $(a,b)$ by the Euclidean algorithm is at most 5 times the number of decimal digits in $a$.  
\end{theorem}

\begin{proof}
  Suppose that $a$ contains $k$ decimal digits and takes $n$ divisions to compute $(a,b)$. We need to show that $n \leq 5k$. Let $r_0 = b, r_1 = a$. Applying Division algorithm repeatedly, we have:
  \[r_0 = r_1q_1+r_2, 0 < r_2 < r_1\]
  \[r_1 = r_2q_2+r_3, 0 < r_3 < r_2\]
  \[\cdots\]
  \[r_{n-1} = r_n \cdot q_n + 0\]

  We can prove that $q_i \geq 1$ for $1 \leq i \leq n-1$ and $q_n \geq 2$. This is because if $q_n$ is 1, then $r_{n-1} = r_n$, which is a contradiction. If $q_i = 0$, then $r_{i-1} = r_{i+1}$, which is also a contradiction.
  We claim that $r_{n-i} \geq F_{i+2}$ for $1 \leq i \leq n-1$. This is true for the base case, where $r_{n-1} \geq F_3 = 2$. 
  \begin{align*}
    r_{n-j} &= r_{n-(j-1)}q_{n-(j-1)} + r_{n-(j-2)}\\
            &\geq r_{n-(j-1)} + r_{n-(j-2)}\\
            &\geq F_{j+1} + F_j = F_j+2
  \end{align*}
  In particular, $a = r_1 \geq F_{n+1}$. Now $10^k > a \geq F_{n+1} \heq \alpha^{n-1}$, where $\alpha = \frac{1+\sqrt5}{2}$ and $n \geq 3$. Taking logarithms and using the fact that $\log \alpha > 1/5$, we get that $n \leq 5k$.
\end{proof}

\begin{exercise}
  Prove that for all $a,m,n \in \N$,
  \[(a^n-1,a^m-1) = a^{(n,m)}-1\]
\end{exercise}
\begin{solution}
  Let $f_n = a^n-1$. Our goal is to prove that $(f_n,f_m) = f_{(n,m)}$. 
  \begin{align*}
    f_n &= a^n-1 \\
        &= a^{n-m}(a^m-1) + a^{n-m} - 1 \\
        &= f_{n-m} + kf_m
  \end{align*}
  So now, when performing Euclid's algorithm to find the GCD, $(f_n,f_m) = (f_{n-m},f_m)$. This occurs recursively, and as we can see it is also performing Euclid's algorithm in the subscript. Hence, $(f_n,f_m) = f_{(n,m)}$.
\end{solution}

\section{Least Common Multiple}

\begin{definition}
  Let $a,b \in \N$. We say that a positive integer $l$ is the least common multiple of $a$ and $b$ if 
  \begin{itemize}
    \item $a$ and $b$ both divide $l$
    \item $c$ is a positive integer divisible by both $a$ and $b$, then $l \leq c$.
  \end{itemize}
  It is generally denoted by $[a,b]$.
\end{definition}

\begin{exercise}
  Prove that $(a,b)[a,b] = ab$.
\end{exercise}
\begin{solution}
  Let $d = (a,b)$ and let $k = ab/d$. Obviously, $k \in \N$. We have to show that $k = [a,b]$. We first prove that $a$ and $b$ both divide $k$. Write 
  \[a = da_1,b=db_1,a_1,b_1 \in \Z\]
  Then,
  $k = \frac{ab}{d} = ab_1 = ba_1$.
  Hence $a$ and $b$ both divide $k$.
  Suppose $c$ is a positive integer divisible by both $a$ and $b$. Write
  \[c = aa', c = bb', a',b' \in \Z\]
  Since $d = (a,b)$, there exists integers $x$ and $y$ such that:
  \[d = ax+by\]
  Then 
  \[\frac{c}{k} = \frac{cd}{ab} = \frac{c(ax+by)}{ab} = \frac{c}{b}x + \frac{c}{a}y = b'x+a'y \in \Z\]
  Hence $k$ divides $c$. Since $c \neq 0$, this implies $k \leq c$. Therefore, by definition, $k = [a,b]$ and $(a,b)[a,b] = ab$.
\end{solution}

\section{Fundamental Theorem of Arithmetic}

\phantom{}

\begin{theorem}[Fundamental Theorem of Arithmetic]
  Every integer $n \geq 2$ can be written as a product of primes and the factorization into primes is unique up to the order of the factors.
\end{theorem}
\begin{proof}
  Let $P(n)$ be the statement that $n$ can be written as a product of primes. $P(2)$ is true, since it is a prime. Let $k \geq 2$ and $P(2), \cdots , P(k)$ is true.
  If $k+1$ is prime, then $P(k+1)$ is immediately true. If it is composite, then $k+1$ is of the form $a \cdot b$ where $2 \leq a,b \geq k$. By induction hypothesis, since $P(a)$ and $P(b)$ is true, $P(k+1)$ is also true.

  Suppose that 
  \[n = p_1p_2..p_r = q_1...q_s\]
  Assume $r \leq s$. Since if $p_1|q_1...q_s$, $p_1 = q_i$ for some $i, 1 \leq i \leq s$. We can cancel this out and keep doing this till we get
  \[1 = q_{i_1}q_{i_2}q_{i_{s-r}}\]
  However, $q_i$ is a prime. This is a contradiction unless $s = r$. Hence uniqueness is proved.
\end{proof}

The largest power $x$ such that $p^x|a$ for a prime $p$ is called the multiplicity of $p$ in $a$.

\chapter{Linear Diophantine Equations}

\section{Linear Diophantine Equations in Two Variables}

Linear Diophantine equations in two variables are of the form:
\[ax+by=c\]

An integer solution exists if and only if $(a,b)|c$. In this case, for any initial solution $x_0,y_0$, the general solution is given by:
\[x = x_0 + \frac{b}{d}n, y = y_0 - \frac{a}{d}n\]
where $d = (a,b)$ and $n \in \Z$. The initial solution $x_0,y_0$ can be found using the Extended Euclidean Algorithm.

\section{Fibonacci Linear Diophantine Equation}
\phantom{}
\begin{theorem}
  Let $k \in \N, c \in \Z$. The linear Diophantine equation:
  \[F_{k+1}x + F_ky = c\]
  is always solvable. If $k$ is even, the complete solution is given by:
  \[x=F_{k-1}c+F_kn, y=-F_kc -F_{k+1}n\]
  If $k$ is odd, the complete solution is given by:
  \[x=-F_{k-1}c + F_kn, y = F_kc - F_{k+1}n\]
\end{theorem}
\begin{proof}
  $(F_k,F_{k+1}) = 1$, so the solution always exists. 
  If $k$ is even, then:
  \[x_0 = F_{k-1}c, y_0 = -F_kc\]
  always solves it (prove it via Cassini's Formula). If $k$ is odd, then:
  \[x_0 = -F_{k-1}c, y = F_kc\]
  always solves it.
\end{proof}

\section{Linear Diophantine Equations in Many Variables}
\phantom{}
\begin{theorem}
  Let $a_1,...,a_k$ be integers, not all zero. The Diophantine equation:
  \[a_1x_1 + ... + a_kx_k = x\]
  has an integer solution if and only if $(a_1,...,a_k)|c$. In that case, there are infinitely many solutions.
\end{theorem}

\begin{exercise}
  Find the complete solution to:
  \[6x+8y+12z=10\]
\end{exercise}
\begin{solution}
  $(6,8,12)  = 2$, and $2|10$, so this equation is solvable.
  Let us consider the ``subequation'' $8y+12z$. Since it is a linear combination of 8 and 12, it must be a multiple of $(8,12) = 4$. So,
  \[8y+12z = 4u\]
  This gives us a new equation:
  \[6x+4u = 10\]
  The general solution of this is given by:
  \[x = 1+2n, u = 1-3n\]
  Substituting $u$ in the first equation,
  \[8x+12y = 4(1-3n)\]
  This equation is also solvable, and we can find it's general solution as well. Since $4=2.8 + (-1).12$, we get $4(1-3n) = (2-6n)8 + (-1-3n)12$. Finally, we can write the complete solution:
  \[x = 1+2n\]
  \[y = 2-6n + 3m\]
  \[z = -1+3n -2m\]
\end{solution}

\begin{exercise}
  Suppose we have a coin system. Let's say we use 4 rupee and 9 rupee coins only. Then which amounts can be exchanged? For instance, 1 rupee cannot be exchanged; 17 rupees can be exchanged. 
\end{exercise}
\begin{solution}
  This problem can be expressed as - what numbers can be expressed as a linear combination of 4 and 9 with positive coefficients.
  In fact, only finitely many integers cannot be exchanged. Generally, if $a,b \in \N$ with $(a,b) = 1$, then the largest integer which cannot be represented as $ax+by$ with $x,y \in \Z, y,y \geq 0$ is $ab-a-b$.

  TODO complete proof
\end{solution}

\chapter{Congruence}

\begin{definition}
  Let $m \in \N$. Two integers $a,b$ are said to be congruent modulo $m$ if $m|a-b$. This is denoted by $a \equiv b \mod m$  
\end{definition}

Congruence has the following properties:

\begin{itemize}
  \item $a \equiv a \mod m$
  \item $a \equiv b \mod m$, then $b \equiv a \mod m$
  \item $a \equiv b \mod m$ and $b \equiv c \mod m$, then $a \equiv c \mod m$
\end{itemize}

As such, it is an equivalence relation. The set of all integers congruent to $a$ modulo $m$ is denoted by $[a]$, and is called the congruence class of $a \mod m$

\begin{lemma}
  Two integers are congruent modulo $m$ if and only if $a,b$ have the same remainder upon division by $m$ 
\end{lemma}
\begin{proof}
Suppose $a$ and $b$ have the same remainder upon division by $m$. Then:
\[a = mq_1 + r\]
\[b = mq_2 + r\]
\[a-b = m(q_1-q_2) + 0\]
\[m|a-b \Rightarrow a \equiv b \mod m\]

Suppose $a \equiv b \mod m$. This means that $a-b = mk, k \in \Z$. By division algorithm, there exist unique integers $q,r$ such that:
\[b = mq+r, 0 \leq r < m\]
\[a = m(q+k) + r\]
Since $0 \leq r < m$, $r$ is the remainder of $a$ upon division by $m$. So, they have the same remainder. Hence proved.
\end{proof}

So, every integer $a$ is congruent to it's remainder $r$ modulo $m$. We call $r$ the least residue of $a$ modulo $m$. Every integer is congruent to exactly one of the least residues $0,1,...,m-1$ modulo $m$. Hence,
\[\Z = \bigcup_{i=0}^{m-1} [i]_m\]

\begin{lemma}
  Suppose $a \equiv b \mod m$ and $x \equiv d \mod m$. Then:
  \begin{itemize}
    \item $a \pm c \equiv b \pm d \mod m$
    \item $ac \equiv bd \mod m$
    \item $a^n \equiv b^n \mod m$
  \end{itemize}
\end{lemma}

\end{document}
