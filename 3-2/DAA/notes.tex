\documentclass[12pt,letterpaper]{article}

%%%%%%%%%%%%
% Includes %
%%%%%%%%%%%%
\usepackage[utf8]{inputenc}
\usepackage[margin=1 in]{geometry}
\usepackage{graphicx}
\usepackage{amsmath}
\usepackage{amsthm}
\usepackage{amsfonts}
\usepackage{parskip}
\usepackage[justification=centering]{caption}
\usepackage[ruled,vlined]{algorithm2e}
\usepackage[hidelinks]{hyperref}

% Formatting
\renewcommand{\baselinestretch}{1.25}

% Theorems and other necessary structures
\theoremstyle{definition}
\newtheorem{definition}{Definition}[section] % Definition
\newtheorem{theorem}{Theorem}[section] % Big result
\newtheorem{corollary}{Corollary}[theorem] % Follows from a theorem
\newtheorem{lemma}[theorem]{Lemma} % Minor result
\newtheorem*{exercise}{Exercise}

\newenvironment{solution}
  {\renewcommand\qedsymbol{$\blacksquare$}\begin{proof}[Solution]}
  {\end{proof}}

% New commands
\newcommand{\R}{\mathbb{R}}
\newcommand{\N}{\mathbb{N}}
\newcommand{\Z}{\mathbb{Z}}

% Title information
\title{Design and Analysis of Algorithms}
\author{2018A7PS0193P}

\begin{document}

\maketitle

\section{Fundamentals}

\begin{definition}
  An algorithm is a well defined computational procedure. It takes an input, does some computation and terminates with output
\end{definition}

To check the correctness of an algorithm, we must check the following characteristics:

\begin{itemize}
  \item Initialization: The algorithm is correct at the beginning
  \item Maintenance : The algorithm remains correct as it runs
  \item Termination : The algorithm terminates in finite time, correctly
\end{itemize}

For this entire course, we must always prove these characteristics when defining any algorithm.

Algorithms are generally defined by a complexity - the time taken to complete the computation on a given input size. There are three ways we could consider this - best case, worst case, or average case.

Complexity is discussed a lot in DSA, so I'm not going to rewrite it here. A quick roundup is:

\begin{itemize}
  \item $O(g(n)) = \{f(n) : \text{there exists } c, n_0 : 0 \leq f(n) \leq c \cdot g(n) \forall n \geq n_0 \}$
  \item $\Omega(g(n)) = \{f(n) : \text{there exists } c, n_0 : 0 \leq c \cdot g(n)\leq f(n) \forall n \geq n_0 \}$
  \item $\Theta(g(n)) = \{f(n) : \text{there exists } c, n_0 : 0 \leq c_2 \cdot g(n)\leq f(n) \leq c_2 \cdot g(n) \forall n \geq n_0 \}$
\end{itemize}

To find complexities in the case of recurrences, we use the \textbf{master method}. Let the recurrence be given by:

\[T(n) = aT\left(\frac{n}{b}\right) + f(n)\]

Here, $a, b \geq 1$. Let $\epsilon$ be a constant. Then:

\begin{enumerate}
  \item If $f(n) = O(n^{\log_ba - \epsilon})$ then $T(n) = \Theta(n^{\log_ba})$
  \item If $f(n) = O(n^{\log_ba})$ then $T(n) = \Theta(n^{\log_ba}\log{n})$
  \item If $f(n) = O(n^{\log_ba + \epsilon})$ then $T(n) = \Theta(f(n))$ provided if $af(n/b) \leq cf(n)$ for some constant $c <1$ and all sufficiently large $n$.
\end{enumerate}

Here, we redo DSA despite it being a prerequisite of the course. This recap has lasted 3 lectures (so far). You should probably just read CLRS, this is a waste. The topics covered are:

\begin{itemize}
  \item Quicksort (and it's average case analysis)
  \item The $\Omega(n\log{n})$ lower bound of comparison sorting
  \item Non-comparison sorting like counting sort, radix sort, etc.
  \item Average case analysis of bucket sort
\end{itemize}

\section{Karatsuba Multiplication}

Karatsuba multiplication is a divide and conquer method to speed up multiplication of large integers. Usually if multiplying two numbers $x$ and $y$ with $n$ digits each takes $O(n^2)$ time, but Karatsuba multiplication does it in $O(n^{1.59})$.

Let us consider the strings in some base $B = 10$. Then, we can write
\[x = x_1 B^m + x_0\]
\[y = y_1 B^m + y_0 \]
where $m = n/2$.
The product will be given by:
\[xy = z_2 B^{2m} + z_1 B^m + z_0\]
where 
\[z_2 = x_1y_1\]
\[z_1 = x_1y_0 + x_0y_1\]
\[z_0 = x_0y_0\]
This seems to need 4 multiplications, but it can in fact be done only in 3, by observing that:
\[z_1 = (x_1 + x_0)(y_1 + y_0) - z_2 - z_0\]

So, we get the algorithm:

\begin{algorithm}[H]
  \SetAlgoLined
  p = KMul(x_1+x_0,y_1+y_0) \\
  x_1y_1 = KMul(x_1,y_1) \\ 
  x_0y_0 = KMul(x_0,y_0) \\
  \Return x_1y_1 \times 10^n + (p-x_1y_1 - x_0y_0)\times 10^{n/2} + x_0y_0
  \caption{KMul(x,y)}
\end{algorithm}

The time complexitu of this is $T(n) = 3T(n/2) + cn$, which gives the aforementioned complexity.

\section{Matrix Multiplication}

Naive Matrix multiplication is $\Theta(N^3)$, since we can express the result $A \cdot B  = C$ as:

\[C_{ij} = \sum_{k=1}^{r}A_{ik} \times B_{kj}\]

We can improve this using a divide-and-conquer approach with \textbf{Strassen's Multiplication}. It has four steps:

\begin{enumerate}
  \item Divide the input matrices $A$ and $B$ and the output matrix $C$ into four $n/2 \times n/2$ submatrices. This takes $\Theta(1)$ time.
  \item Create 10 matrices $S_1,S_2,...S_{10}$, each of which is of size $n/2 \times n/2$ and is the sum or difference of two matrices created in step 1.
  \item Using these submatrices, we can recursively compute seven matrix products $P_1,P_2,...P_7$, each of which is $n/2$.
  \item Compute the desired submatrices $C_{11},C_{12},C_{21},C_{22}$ by adding and subtracting various combinations of the $P_i$ matrices. We can compute all four in $\Theta(N^2)$ time.
\end{enumerate}

The details of this can be seen on page 80 of CLRS, but the running time recurrence will be given by:

\[T(n) = \begin{cases}
  \Theta(1) & \text{if } n = 1 \\
  7T(n/2) + \Theta(n^2) & \text{if } n > 1
\end{cases}\]

By master method, this is $T(n) = \Theta(n^{\log7})$

\section{Polynomial Multiplication}

Polynomials are functions of the form:

\[f(x) = a_0 + a_1x + a_2x^2 + ... a_{n-1}x^{n-1}\]

One way to express this is as a vector of coefficients  - this is called the \textbf{Coefficient form}. This form also allows us to evaluate $f(x)$ in $O(n)$ using Horner's rule, where we express the polynomial as:

\[a_0 + x(a_1 + x(a_2 + ...x(a_n-1)...))\]

Another way to express this is using the \textbf{point value form} , where we express it as $n$ point of the form $(x_i,f(x_i))$. This point value form uniquely identifies a polynomial. Generally finding the point value form from a coefficient form would take $\Theta(N^2)$ time , but with FFT we can do it in $O(N\log N)$.

The process of getting the coefficient form from the point value form is known as $\textbf{interpolation}$.  We can do this in $O(n^3)$ using Gaussian Elimination, or in $O(n^2)$ with Lagrange Interpolation.

Generally, the multiplication of polynomials takes $\Theta(N^2)$. However, we can do this much faster using \textbf{Fast Fourier Transform}.

\subsection{Fourier Transform}

\subsubsection{Discrete Fourier Transform}

From now on, $w_n^k$ will denote the $k^{th}$ solution of $x^n=1$, i.e. the $n^{th}$ root of unity. $w_n$ will denote the principal $n^{th}$ root of unity. Remember the following properties:

\begin{enumerate}
  \item $w_n^k = e^{\frac{2k\pi i}{n}}$
  \item $w_{dn}^{dk} = w_{n}^k$
  \item $w_{n}^{n/2} = w_2 = -1$
  \item If $n >0$ is even, then the squares of the $n$ complex $n^{th}$ roots of unity are the $n/2$ complex $n/2$th roots of unity. (Halving Lemma)
  \item $\sum_{j=0}^{n-1} (w_{n}^{k})^j = 0$ (Summation Property)
\end{enumerate}

We call the vector $y = (y_0,y_1,...y_{n-1})$ the \textbf{discrete Fourier Transform}  of the polynomial $A$ if $y_k = A(w_n^k)$.

In our case of polynomial multiplication, we will pad the polynomials with zero to the closest power of 2, such that it is at least double the size, and then compute DFT with this new size. This will help us since we need $2n$ point values to find the coefficient form of the product.

\subsubsection{Fast Fourier Transform}

FFT consists of three parts:

\begin{enumerate}
  \item Evaluation, where we find the DFT in $O(n\log n)$
  \item Pointwise Multiplication of the two DFTs
  \item Interpolation or the inverse FFT, where we find the coefficient form in $O(n \log n)$.
\end{enumerate}

Consider the following polynomials:

\[A(x) = a_0x^0 + a_1x^1 + ... + a_{n-1}x^{n-1}\]
\[A_0(x) = a_0x^0 + a_2x^1 + ... + a_{n-2}x^{n/2-1}\]
\[A_1(x) = a_1x^0 + a_3x^1 + ... + a_{n-1}x^{n/2-1}\]

It is easy to see that:

\[A(x) = A_0(x^2) + xA_1(x^2)\]

These polynomials have only half as many coefficients as the polynomial $A$. So, if we can compute $DFT(A)$ from $DFT(A_1)$ and $DFT(A_0)$ in linear time, we would be able to do this in $O(n \log n)$ (direct from master method).

We find that we can do this with the equations:

\[y_k = y_k^0 + w_n^k y_k^1\]
\[y_{k+n/2} = y_k^0 - w_n^k y_k^1\]

Here, $k \in [0,n/2-1]$. The first equation is clear, and comes directly from the formula. The proof for the second is as follows:

\begin{align*}
  y_{k+n/2} &= A(w_n^{k+n/2}) \\
            &= A_0(w_n^{2k+n}) + w_n^{k+n/2}A_1(w_n^{2k+n}) \\
            &= A_0(w_n^{2k} w_{n}^n) + w_n^k w_n^{n/2} A_1(w_n^{2k} w_n^{n}) \\
            &= A_0 (w_n^{2k}) - w_n^kA_1(w_n^{2k}) \\
            &= y_k^0 - w_n^k y_k^1
\end{align*}

As such, we have found the DFT of the polynomial in $O(n \log n)$ time.

After performing the pointwise multiplication of the DFT of our polynomials $A$ and $B$, we have to interpolate to find the coefficient form of our resulting polynomial $C$.

\textit{TODO: Add interpolation with Vandermonde matrix, can be seen on cp-algorithms} 

In short, the formula is:
\[a_k = \frac{1}{n} \sum_{j=0}^{n-1} y_j w_n^{-kj}\]

The coefficients can be found via the same divide and conquer algorithm as in the direct FFT, except we use $w_n^{-k}$ instead of $w_n^k$.

I know I've done a poor job of explaining it, but I don't want to copy paste the entirety of the cp-algorithms post here. Check it out.

\section{Greedy algorithms}

Greedy algorithms involve making a sequence of choices where each looks best at the moment. It is making the locally optimal choice in the hope that it leads to a globally optimal solution. However, this may not always be the case. For this to work, we need the following properties:

\begin{itemize}
  \item \textbf{Greedy choice property:} we can assemble a globally optimal solution by making locally optimal (greedy) choices.
  \item \textbf{Optimal Substructure} : A problem is said to have optimal substructure if an optimal solution to the problem contains within it optimal solutions to subproblems. 
\end{itemize}

\subsection{Activity Selection Problem}

Consider a set $S = \{1,2,3,..n\}$ of $n$ activities that can happen one activity at a time. Activity $i$ takes place during interval $[s_i,f_i)$. Activities $i$ and $j$ are compatible if $[s_i,f_i)$ and $[s_j,f_j)$ don't overlap. Our goal is to select the maximum size subset of mutually comparable activities. 

To solve, we can assume that activities are in increasing order of their finishing time. If not, then sort it in $O(n \log n)$. Doing this, we can choose them greedily, picking an activity whenever we are free.

\subsection{Fractional Knapsack Problem}

A thief robbing a store finds $n$ items. The $i^{th}$  item is worth $v_i$ dollars and weighs $w_i$ pounds. The thief wants to take as valuable a load as possible, but he can carry at most $W$ pounds in his knapsack. Which items should he take?

If the thief can carry fractions of items, he can solve it greedily - this is called the fractional knapsack problem. Otherwise, if he can either take or leave an item (the 0-1 knapsack problem), then it needs to be solved by dynamic programming.

In the fractional knapsack problem, we can greedily choose the items with the largest value to weight ratio.

\subsection{Huffman Coding}

Huffman coding is a greedy algorithm that constructs an optimal prefix code.

Say we are given a text, along with the frequencies of each character in the text. Obviously, we want the most frequent character to take up the minimum number of bits, to minimize the total size. We can make the same arguments for the other characters in decreasing order of frequencies. For instance, if the character `a' has the most frequency, we may represent it by the bit 0, and the character `x' (which is next in the order of frequency) as 10 and so on. We have to design these so that there is no ambiguity when decoding the Huffman code. This means no code can be the prefix of any other code!

We can represent this encoding as a binary tree where going left corresponds to adding the character `0' to the code, and moving right corresponds to adding the character `1' to the code. Each character is a leaf in this binary tree, and they will definitely not be prefixes of one another.

The algorithm for Huffman Coding creates this tree. Say we are given a set $C$ of characters, along with theire frequencies. The algorithm is as follows:

\begin{algorithm}[H]
  \SetAlgoLined
  \KwResult{A prefix tree for Huffman Codes}
  $n = |C|$ \\
  $Q = C$ \\
  \For{$i \gets1$ \KwTo $n-1$} {
    Allocate new node $z$ \\
    $z.left = x =$ EXTRACT\_MIN(Q)$ \\
    $z.right = y =$ EXTRACT\_MIN(Q)$ \\
    $z.freq = x.freq+y.freq$ \\ 
    INSERT($Q.z$) \\
  }
  return EXTRACT\_MIN(Q)
  \caption{Huffman Coding}
\end{algorithm}

If we use a heap, we can do this in $O(n \log n)$. It can actually be faster using van Emde Boas Tree, which would make it $O(n \log \log n)$

In a Huffman coding, the average bit length is given by:
\[\frac{\sum_{c \in A} |H_c| f_c}{\sum_{c \in S} f_c}\]

where $A$ is the alphabet, $H_c$ is the Huffman code for $c$, and $f_c$ is the frequency of $c$.

\section{Matroids}

A \textbf{matroid}  is an ordered pair $M = (S,I)$ satisfying the following conditions:

\begin{itemize}
  \item $S$ is a finite set
  \item $I$ is a non empty family of subsets of $S$ called the independent subsets of $S$, such that if $B \in I$ and $A \subseteq B$, then $A \in I$. This is the Hereditary Property.
  \item If $A \in I$, $B \in I$, and $|A| \leq |B|$ then there exists some element $x\in B-A$ such that $A \cup \{x\} \in I$. This is the exchange property.
\end{itemize}

In more simple terms, the matroid gives a classification of each subset of $S$ to be independent or dependent. The empty set is always independent and any subset of an independent set is independent. If an independent size $A$ has smaller size than $B$, then there exists some element in $B$ that can be added into $A$ without loss of independency.

Given a matroid $M$, we call an element $x \notin A$ an \textbf{extension} of $A \in I$ if we can add $x$ to $A$ while preserving its independence, i.e. $A \cup \{x\} \in I$. 

The graphic matroid $M_G = (S_G, I_G)$ is defined as follows:

\begin{itemize}
  \item The set $S_G$ is the set of edges in the graph $G$
  \item If $A$ is a subset of $E$ (edges), then $A \in I_G$ if and only if $A$ is acyclic. That is, a set of edges $A$ is independent if and only if the subgraph $G_A = (V,A)$ forms a forest
\end{itemize}

\begin{theorem}
  If $G = (V,E)$ is an undirected graph, then $M_G = (S_G,I_G)$ is a matroid.
\end{theorem}

\begin{theorem}
  All maximal independent subsets in a matroid have the same size.
\end{theorem}

This is obvious given that if a maximal independent subset had a smaller size, it could be extended using the exchange property.

A matroid $M = (S,I)$ is said to be \textbf{weighted} if it is associated with a strictly positive weight function $w(x)$ for all $x \in S$. $w(A)$ is defined as :

\[w(A) = \sum_{x \in A} w(x)\]

The independent set with maximum $w(A)$ is called an \textbf{optimal subset} of a matroid. An optimal subset is always a maximal independent subset.

Let us consider the Minimum Spanning Tree problem, where we seek the subset of edges that connects all the vertices together and has minimum total length. This is like finding the optimal subset of a weighted matroid $M_G$ where weight function $w'(e) = w_0 - w(e)$, where $w(e)$ is the weight of the edge and $w_0$ is some constant greater than all the weights. A greedy algorithm for a weighted matroid is:

\begin{algorithm}[H]
  \SetAlgoLined
  $A = \phi$ \\
  sort $M.S$ into monotonically decreasing order of weight $w$  \\
  \For{each $x \in M.S$}{
    \If{$A \cup \{x\} \in M.I$}
    {
      $A = A \cup \{x\}$\\
    }
  }
  \Return A \\
  \caption{Greedy(M,w)}
\end{algorithm}

Notice, that this is basically Kruskal's algorithm for finding a minimum spanning tree. In graph theory terms, we are sorting all the edges in increasing order of edge weight, and choosing these edges one by one as long as they do not form a cycle (not in the independent set). The check for cycle can be done with a disjoint set union, in this case.

\begin{lemma}[Greedy Choice Property]
  Consider $M = (S,I)$ with weight function $w$. Let $S$ be sorted in decreasing order. Consider $x$, the first element of $S$ such that $\{x\} $ is independent. If this exists then there exists an optimal subset $A$ containing $x$.
\end{lemma}

\begin{lemma}
  Let $M = (S,I)$ be any matroid. If $x$ is an element of $S$ that is an extension of some independent subset $A$ of $S$, then $x$ is also an extension of $\phi$. 
\end{lemma}

\begin{corollary}
  Let $M = (S,I)$ be any matroid. If $x$ is an element of $S$ such that $x$ is not an extension of $\phi$, then $x$ is not an extension of any independent set $A$ of $S$. 
\end{corollary}

These lemmas tell us that at any point, choosing the minimum is optimal, as long as it does not create a cycle.

\begin{lemma}[Optimal substructure property]
  Let $x$ be the first element of $S$ chosen by GREEDY for the weighted matroid $M = (S,I)$. We can reduce this problem to $M' = (S',I')$, such that:
  \begin{itemize}
    \item $s' = \{y \in S : \{x,y\} \in I\}$
    \item $I' = \{B \subseteq S - \{x\} : B \cup \{x\} \in I\}$
  \end{itemize}
\end{lemma}

This lemma tells us that removing the edge $x$ with minimum edge weight yields a new matroid for us to continue our greedy choices on.

From the above lemmas, we can be sure that our greedy solution is optimal.

\section{0-1 Knapsack}

No notes for this, it's too simple. $O(N*W)$ algorithm. For general information, there are lots of faster algorithms if you add some extra constraints.

The transitions are:

\[M(i,w) = \max\{M(i-1,w), M(i-1,w-w_i) + p_i\}\]

\section{Travelling Salesman Problem}

Consider we have a graph, where every edge between vertices $i$ and $j$ has some weight $c_{ij}$. Our goal is to find a path where we start from one city, visit every other city and return to the same one again, in the cheapest manner. This is like finding a Hamiltonian Cycle in the graph.

Let $g(i,S)$ be the length of the shortest path starting at vertex $i$, going through all the vertices in $S$ and terminating at 1. Then the following equations are obvious:

\[g(1,V-\{1\}) = \min_{2 \leq k \leq n} \{c_{1k} + g(k,V-\{1,k\})\}\]
\[g(i,S) = \min_{j \in S} \{c_{ij} + g(j,S-\{j\})\}\]

From this, we can design the TSP algorithm:

\begin{algorithm}[H]
  \SetAlgoLined
  \For{$i = 2$ to $n$} {
    $g(i,\phi) = c_{i1}$
  }
  \For{$k = 1$ to $n-2$}{
    \For{$i=2$ to $n$} {
      \For{$S \subseteq V - \{i,1\}$ with $|S| = k$}{
        $g(i,S) = \min_{j \in S} \{c_{ij} + g(j, S-\{j\})\}$
      }
    }
  }
  $g(1,V-\{1\}) = \min_{j \in S} \{c_{1i} + g(i,V-\{1,i\})\}$ \\
  \Return $g(1,V-\{1\})$
  \caption{TSP($V,c_{ij}$)}
\end{algorithm}

The time complexity of this is $T(n) = \Theta(n^2 \cdot 2^n)$ and space complexity $\Theta(n2^n)$.

\section{Matrix Chain Multiplication}

If we are given a sequence of matrices, $A,B,C$ of size $u \times v$, $v \times w$, $w \times z$ respectively. This gives us two ways to multiple the matrix : 

\begin{itemize}
  \item $(A \times B) \times C$ : Takes $u \times v \times w + u \times w \times z$ steps
  \item $A \times (B \times C)$ : Takes $u \times v \times z + v \times w \times z$ steps
\end{itemize}

Our goal is to find the order of multiplication that would take the minimum number of steps.

One way to do this could be brute force, where we try every order of multiplication. This problem is equivalent to finding the number of ways to parenthesize an expression of $n$ matrices. This can be expression by the recursion:

\[P(n) = \begin{cases}
1 & \text{if } n = 1 \\
\sum_{k=1}^{n-1} P(k) \times P(n-k) & \text{otherwise}
\end{cases}
\]
This is, in fact, the $n-1$ Catalan number $C(n-1)$, where:
\[C(n) = \frac{1}{n+1} {2n \choose n}\]

A more efficient approach to solve this is DP. Let us assume every matrix $A_i$ has the dimensions $p_{i-1} \times p_i$. Then we can use the following DP:

\begin{algorithm}[H]
  \SetAlgoLined
  $n$ = length[$p$] - 1 \\
  \For{i = 1 to n} {
    $m[i][i] = 0$
  }
  \For{l = 2 to n} {
    \For{i=1 to n-l+1} {
      $j=i+l-1$ \\
      $m[i][j] = \infty$ \\
      \For{k=i to j-1} {
        $q = m[i][k] + m[k+1][j] + p_{i-1}p_kp_j$ \\
        \If{$q < m[i][j]$}{
          $m[i][j] = q$ \\
          $s[i][j] = k$ 
        }
      }
    }
  }
  \caption{Matrix-Chain-Order(p)}
\end{algorithm}

This is a standard DP by length. First we realize that in any range $A_{i..j}$ we can split the range between $A_k$ and $A_{k+1}$, in such a way that the parenthesization of the prefix $A_{i..k}$ is optimal. This is because if there was a less costly way to parenthesize $A_{i..k}$, we could replace it with that and reduce the total cost. From this, we can split a range into two parts - a prefix and a suffix, and solve recursively on it. To combine two solutions, we would use the equation:
\[m[i,j] = m[i,k] + m[k+1,j] + p_{i-1} p_k p_j\]
By memoizing these values, we can generate this DP.

\section{Longest Common Subsequence}

This is very common and standard, so I'm only writing the transitions here.

\[
  LCS(X_i,Y_j) = \begin{cases}
    0 & \text{ if } i = 0 \text{ or } j = 0 \\
    LCS(X_{i-1},Y_{j-1})+1 & x_i = y_j \\
    \max \{LCS(X_i, Y_{j-1}), LCS(X_{i-1},Y_j)\} & x_i \neq y_j
  \end{cases}
\]

This only gives lengths, but the exact LCS can be found by moving backwards on the DP table.

Interesting fact is that the longest palindromic subsequence in a string is the LCS of the string and it's reverse.

\section{Optimal Binary Search Trees}

Suppose that we are designing a program to translate text from English to French. For each occurrence of English word in the text, we need to look up it's French equivalent. This can be done using a binary tree, and could ensure $O(\log n)$ time. However, words occur at different frequencies, so there could be a different total cost of search given a text. So, we want a optimal binary search tree.

Formally, we are given $n$ keys $K = k_1,k_2,...,k_n$ in sorted order and wish to build a BST on these keys. For each $k_i$, we have a $p_i$ probability that the search will be for $k_i$. Some searches may be for values not in $K$, so we also have $n+1$ dummy keys $d_0,d_1,...d_n$. In particular, $d_i$ represents values between $k_i$ and $k_{i+1}$. Each of these have a probability $q_i$. Of course,

\[\sum_{i=1}^n p_i + \sum_{i=0}^n q_i = 1\]

The expected cost in a tree T is given by:

\[E[T] = \sum_{i=1}^n (depth_{T}(k_i) + 1) \cdot p_i + \sum_{i=0}^n (depth_{T}(d_i) + 1)\cdot q_i \]
\[E[T] = 1 + \sum_{i=1}^n depth_{T}(k_i) \cdot p_i + \sum_{i=0}^n depth_{T}(d_i) \cdot q_i \]

Any non-leaf subtree of our BST must contain keys in a continuous range $k_i \cdots k_j$. Each subtree must be optimal, since if it were not, then we could replace it with the more optimal version and create a better tree. This creates our subproblems to divide into and DP on. Let $e[i,j]$ be the expected cost for an optimal BST of keys $k_i,\cdots,k_j$, and let $w[i,j]$ be such that:

\[w[i,j] = \sum_{v=i}^j p_v + \sum_{v=i-1}^j q_v\]

Then if $k_r$ is root,
\begin{align*}
  e[i,j] &=  p_r + e[i,r-1] + w[i,r-1] + e[r+1,j] + w[r+1,j] \\
         &= e[i,r-1] + e[r+1,j] + w[i,j]
\end{align*}

Hence our goal becomes choosing $r$ such that it minimizes $e[i,j]$. This gives us the following algorithm

\begin{algorithm}[H]
  \SetAlgoLined
  \For{i = 1 to n+1} {
    $e[i,i-1] = w[i,i-1] = q_i-1$ \\
  }
  \For{l = 1 to n} {
    \For{i = 1 to n-l+1} {
      $j = i-l+1$ \\
      $e[i,j] = \infty$ \\ 
      $w[i,j] = w[i,j-1] + p_j + q_j$ \\
      \For{r = i to j} {
        $t = e[i,r-1] + e[r+1,j] + w[i,j]$ \\
        \If{$t < e[i,j]$} {
          $e[i,j] = t$ \\
          $root[i,j] = r$
        }
      }
    }
  }
  \caption{Optimal-BST(p,q,n)}
\end{algorithm}

\section{Flow Shop Scheduling}

Consider $n$ jobs each having $m$ tasks $T_{1i}, T_{2i}, ... , T_{mi}$ for $1 \leq i \leq n$ where $T_{ji}$ can be executed on processor $p_j$ only. A processor cannot execute two tasks at a time, and $T_{2i}$ cannot be executed before $T_{1i}$.

This is essentially a scheduling problem. As is common in scheduling, we have two variants - preemptive and non-preemptive.

For a given schedule, $f_i(S)$ is the time taken to complete a job $i$. Then the finish time of a schedule $S$ is given by:
\[F(S) = \max_{1 \leq i \leq n} f_i(S)\]
Our goal is to get the optimal non-preemptive schedule, which would have minimum $F(S)$.

This is difficult to solve for $m > 2$, so let us solve for $m=2$. Let us simplify the notation by using $a_i$ for $t_{1i}$ and $b_i$ for $t_{2i}$. This can be represented by the matrix

\[\begin{bmatrix}
  a_1 & a_2 & a_3 & a_4 & a_5 \\
  b_1 & b_2 & b_3 & b_4 & b_5
\end{bmatrix}\]

When $m=2$, we find that there is nothing to gain by using different schedules for $A$ and $B$. This is because before executing $b_i$, we need to execute $a_i$. So, after finishing the first task for $A$, we would want to immediately start the task for $B$, otherwise we would be wasting time.

Hence, we want a single optimal permutation. We find that in the optimal permutation, given the first job in the permutation, the remaining permutation is optimal. This is because if it were not, we could rearrange it to get the optimal permutation and only improve.

Let $g(S,t)$ be the length of the optimal schedule for the subset of jobs $S$ under the assumption that the processor 2 is not available until time $t$. We want $g(\{1,2,3,\cdots, n\},0)$.
\[g(\{1,2,3,\cdots,n\},0) = \min_{1 \leq i \leq n} \{a_i + g(\{1,2,3,\cdots,n\} - \{i\},b_i)\}\]

Here, we assume that $g(\phi,t) = t$ and $a_i \neq 0$.

Generalizing this,
\[g(S,t) = \min_{i \in S} \{a_i + g(S-\{i\},b_i + \max\{t-a_i,0\})\}\]

If $i$ and $j$ are the first two jobs of the schedule and $B$ is not available for time $t$,

\begin{align*}
  g(S,t) &= a_i + g(S- \{i\},b_i + \max \{t-a_i,0\}) \\
         &= a_i + a_j + g(S-\{i,j\},b_j + \max \{b_i + \max\{t-a_i,0\}-a_j,0\}) \\
\end{align*}
Now, let
\begin{align*}
  t_{ij} &= b_j + \max \{b_i + \max \{t-a_i,0\} - a_j,0\} \\
         &= b_j + b_i -a_j + \max \{ \max \{t-a_i,0\}, a_j - b_i \} \\ 
         &= b_j + b_i - a_j + \max \{t-a_i,0,a_j-b_i\} \\
         &= b_j + b_i - a_j - a_i + \max \{t,a_i,a_j+a_i -b_i\}
\end{align*}

From the above discussion,
\[g(S,t) = a_i + a_j + g(S-\{i,j\},t_{ij})\]

If $i$ and $j$ are interchanged, we get
\[g'(S,t) = a_j + a_i + g(S-\{j,i\},t_{ji})\]

Hence, we can see that
\[g(S,t) < g'(S,t) \Leftrightarrow t_{ij} \leq t_{ji}\]

This should hold for all $t$, so after some reduction, 
\[\min\{a_i,b_i\} \geq \min \{a_j,b_j\}\]

So, if $\min\{a_1,a_2,...,a_n,b_1,b_2,...,b_n\}$ is $a_i$, then $i$ should be the first job. If $\min\{a_1,a_2,...,a_n,b_1,b_2,...,b_n\} = b_j$ then $j$ should be the last job.

\section{Flows}

A \textbf{flow network} is a directed graph $G = (V,E)$ such that:

\begin{itemize}
  \item Every edge $(u,v) \in E$ has a non-negative capacity $c(u,v)$.
  \item There are two distinguished vertices, a source $s$ and a sink $t$.
  \item For each vertex $v \in V$ there exists a path from $s$ to $v$ to $t$.
  \item Self loops are not allowed.
  \item No reverse edges
  \item If $(u,v) \notin E$, $c(u,v) = 0$
  \item The graph is connected.
\end{itemize}

The \textbf{flow} in a graph is a real valued function $f : V \times V \rightarrow \R$ such that:

\begin{itemize}
  \item $\forall u, v \in V, 0 \leq f(u,v) \leq c(u,v)$ (Capacity constraint)
  \item $\forall y \in V - \{s,t\}, \sum_{v \in V} f(v,u) = \sum_{v \in V} f(u,v)$ (Flow conservation)
\end{itemize}

The \textbf{network flow} is defined as:
\[|f| = \sum_{v \in V} f(s,v) - \sum_{v \in V} f(v,s)\]
Typically, no edge enters the source so $\sum_{v \in V} f(v,s) = 0$. However, this definition will be more important when discussing residual networks.

When solving problems with flow, we need to design a flow network. Doing this is at times trivial (see e.g. in CLRS) and at times not (see any Codeforces flow problem). For one, sometimes it may be natural to have antiparallel edges in a graph while modelling - however this is not allowed in flow networks. Remove it by  breaking any one of the edges $(u,v)$ into two - $(u,v'), (v',v)$. Both of these new edges will have the same capacity.

Another issue is when handling a network that could have multiple sources or multiple sinks. In that case, we create supersources and supersinks, nodes which have edges with infinite capacity to other sources or from other sinks.

The \textbf{maximum flow problem} requires us to find a flow of maximum value in the graph. Before looking at the algorithm for this, we need to look at some preliminary concepts.

The \textbf{residual graph} $G_f$ represents the flow of $f$ on $G$ as well as how we can change this flow. The edges of $G$ that are in $G_f$ are those that can admit more flow, and have a residual capacity given by:
\[c_f(u,v) = c(u,v) - f(u,v)\]
The residual network also has extra edges. In order to represent a possible decrease in the flow $f(u,v)$ we place an edge $(v,u)$ in $G_f$ with residual capacity $c_f(v,u) = f(u,v)$. It will admit flow in the opposite direction to $(u,v)$, allowing the flow on an edge to be decreased.

If $f$ is a flow in $G$ and $f'$ is a flow in the corresponding residual network $G_f$, we define $f \uparrow f'$ to be the \textbf{augmentation of flow} $f$ by $f'$, given by:

\[(f \uparrow f')(u,v) = \begin{cases}
  f(u,v) + f'(u,v) - f'(v,u) & (u,v) \in E \\
  0 & \text{otherwise}
\end{cases}\]

We can prove that $|f \uparrow f'| = |f| + |f'|$.

An \textbf{augmenting path} is a simple path from $s$ to $t$ in the residual network $G_f$. We may increase the flow on an edge $(u,v)$ of an augmenting path by up to $c_f(u,v)$. Hence, the amount by which we can increase the flow of each edge in an augmenting path $p$, called the residual capacity of $p$, is the minimum capacity $c_f(u,v)$ in the augmenting path. If we augment the flow $f$ by this amount, we will get another flow, whose value is closer to the maximum.

Now we can define the algorithm to find the maximum flow in a path, called the \textbf{Ford Fulkerson Method} .

\begin{enumerate}
  \item Initialize the flow $f$ to 0
  \item While an augmenting path $p$ exists in the residual network $G_f$, augment the flow $f$ along that path $p$.
\end{enumerate}

The Ford Fulkerson method in fact refers to a class of algorithms, which have the same basic approach but find the augmenting path in different ways. It's complexity is $O(EF)$, where $F$ is the maximal flow in the network. For more specifics, look into algorithms like Dinic, Edmonds-Karp, etc. Remember that is can be difficult to design a worst case for many flow algorithms.

A \textbf{cut} $(S,T)$ of a graph is a partition of $V$ into two disjoint sets $S$ and $T$ such that $s \in S$ and $t \in T$. The net flow across a cut is given by:
\[f(S,T) = \sum_{u \in S}\sum_{v \in T} f(u,v) - \sum_{u \in S} \sum_{v \in T} f(v,u)\]
The capacity of the cut is:
\[c(S,T) = \sum_{u \in S} \sum_{v \in T} c(u,v)\]
The minimum cut of a network is the cut whose capacity is minimum across all cuts.

\begin{theorem}[Max Flow Min Cut Theorem]
  If $f$ is a flow in a flow network $G=(V,E)$ with source $s$ and sink $t$, then the following conditions are equivalent:
  \begin{itemize}
    \item $f$ is a maximum flow in $G$
    \item The residual network $G_f$ contains no augmenting paths
    \item $|f| = c(S,T)$ for some cut $(S,T)$ in $G$
  \end{itemize}
\end{theorem}


\section{Problems}

\begin{exercise}
  Tile a $n \times n$ chessboard with one tile missing using L shaped triominoes, where $n$ is a power of 2.
\end{exercise}
\begin{solution}
  This can be done using divide and conquer. In the case that $n = 2$, it is trivial to fill with the triomino.
  In any other case, we can divide the chessboard into 4 equal subsquares. We then place a $L$ shaped tile such that it does not cover the subsquare containing the missing cell. Now the problem can be solved recursively, since each of the subsquares now have a missing square.
\end{solution}

\begin{exercise}
  Solve the recurrence $T(n) = T(n/3) + T(2n/3) + 1$.
\end{exercise}
\begin{solution}
  We guess that $T(n) = O(n)$ and substitute accordingly. $T(n) \leq cn-d$. Then:
  \begin{align*}
    T(n) & \leq c(\frac{n}{3}) - d + c(\frac{2n}{3})  - d + 1 \\ 
         &\leq cn -2d + 1 \\
         &\leq cn-d
  \end{align*}
  The above holds as long as $d \geq 1$.

  As for the base case, if $T(1) = 1$, we can choose a suitably large value for $c$ and $d$ for it to be true.
\end{solution}

\begin{exercise}
  Solve the recurrence $T(n) = T(n/3) + T(2n/3) + cn$
\end{exercise}
\begin{solution}
  At each level, we can see that the sum of the ``work done''  is $cn$, up till some point. After that, since the right subtree would be doing more work, it would go deeper than the left subtree. 

  First let us consider a lower bound. The lower bound on the height of the tree is obviously $\log_3n$, using the leftmost path. So, considering a full binary tree with that height,
  \[T(n) \geq n \log_3n\]

  Now to consider the deepest path, which would be $\log_{3/2}n$. This tells us that:
  \[T(n) \leq n \log_{3/2}n\]

  From this, we find that $T(n) = \Theta(n \log n)$.
\end{solution}

\begin{exercise}
  How do you use Strassen's algorithm when $n$ is not a power of 2? 
\end{exercise}
\begin{solution}
  You can pad the matrices $A$ and $B$ with 0 until it is a power of 2. This won't affect the complexity since $N > 2n$, where $N$ is the new dimension.
\end{solution}

\begin{exercise}
  How quickly can you multiply a $kn \times n$ matrix by a $n \times kn$ matrix, using Strassen's algorithm? What if the input is reversed?  
\end{exercise}
\begin{solution}
  If we consider the first case, we can see that the $kn \times n$ matrix is composed of $k$ $n \times n$ matrices. We can say the same for $n \times kn$. Therefore, we can write both matrices as $k \times 1$ and $1 \times k$, by assuming each $n \times n$ matrix as a single element. So the running time will be:
  \[T(kn \times n, n \times kn) = k^2 T(n \times n, n \times n)\]
Using Strassen's algorithm, 
\[T(kn \times n, n \times kn) = k^2 n^{\log 7}\]

In the second case, we find that using the same arguments and multiplying a $1 \times k$ and a $k \times 1$ matrix, we would also need to do $k$ multiplications and $k-1$ additions. Therefore,
\[T(n \times kn, kn \times n) = kT(n \times n) + O(k)\]
Which equals
\[T(n \times kn) = k n^{\log 7}\]
\end{solution}

\begin{exercise}
  What are the transitions for the Josephus problem? 
\end{exercise}
\begin{solution}
  The transitions are:
  \[J(n,k) = (J(n-1,k) + k - 1)\%n + 1\]
  \[J(1,k) = 1\]
  $J(n,k)$ is the number of the person who would survive if there were $n$ people and every $k^{th}$ person was killed.
\end{solution}

\end{document}
